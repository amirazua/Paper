
%% bare_conf.tex
%% V1.3
%% 2007/01/11
%% by Michael Shell
%% See:
%% http://www.michaelshell.org/
%% for current contact information.
%%
%% This is a skeleton file demonstrating the use of IEEEtran.cls
%% (requires IEEEtran.cls version 1.7 or later) with an IEEE conference paper.
%%
%% Support sites:
%% http://www.michaelshell.org/tex/ieeetran/
%% http://www.ctan.org/tex-archive/macros/latex/contrib/IEEEtran/
%% and
%% http://www.ieee.org/

%%*************************************************************************
%% Legal Notice:
%% This code is offered as-is without any warranty either expressed or
%% implied; without even the implied warranty of MERCHANTABILITY or
%% FITNESS FOR A PARTICULAR PURPOSE! 
%% User assumes all risk.
%% In no event shall IEEE or any contributor to this code be liable for
%% any damages or losses, including, but not limited to, incidental,
%% consequential, or any other damages, resulting from the use or misuse
%% of any information contained here.
%%
%% All comments are the opinions of their respective authors and are not
%% necessarily endorsed by the IEEE.
%%
%% This work is distributed under the LaTeX Project Public License (LPPL)
%% ( http://www.latex-project.org/ ) version 1.3, and may be freely used,
%% distributed and modified. A copy of the LPPL, version 1.3, is included
%% in the base LaTeX documentation of all distributions of LaTeX released
%% 2003/12/01 or later.
%% Retain all contribution notices and credits.
%% ** Modified files should be clearly indicated as such, including  **
%% ** renaming them and changing author support contact information. **
%%
%% File list of work: IEEEtran.cls, IEEEtran_HOWTO.pdf, bare_adv.tex,
%%                    bare_conf.tex, bare_jrnl.tex, bare_jrnl_compsoc.tex
%%*************************************************************************

% *** Authors should verify (and, if needed, correct) their LaTeX system  ***
% *** with the testflow diagnostic prior to trusting their LaTeX platform ***
% *** with production work. IEEE's font choices can trigger bugs that do  ***
% *** not appear when using other class files.                            ***
% The testflow support page is at:
% http://www.michaelshell.org/tex/testflow/



% Note that the a4paper option is mainly intended so that authors in
% countries using A4 can easily print to A4 and see how their papers will
% look in print - the typesetting of the document will not typically be
% affected with changes in paper size (but the bottom and side margins will).
% Use the testflow package mentioned above to verify correct handling of
% both paper sizes by the user's LaTeX system.
%
% Also note that the "draftcls" or "draftclsnofoot", not "draft", option
% should be used if it is desired that the figures are to be displayed in
% draft mode.
%
\documentclass[conference, compsoc]{IEEEtran}
% Add the compsoc option for Computer Society conferences.
%
% If IEEEtran.cls has not been installed into the LaTeX system files,
% manually specify the path to it like:
% \documentclass[conference]{../sty/IEEEtran}





% Some very useful LaTeX packages include:
% (uncomment the ones you want to load)


% *** MISC UTILITY PACKAGES ***
%
%\usepackage{ifpdf}
% Heiko Oberdiek's ifpdf.sty is very useful if you need conditional
% compilation based on whether the output is pdf or dvi.
% usage:
% \ifpdf
%   % pdf code
% \else
%   % dvi code
% \fi
% The latest version of ifpdf.sty can be obtained from:
% http://www.ctan.org/tex-archive/macros/latex/contrib/oberdiek/
% Also, note that IEEEtran.cls V1.7 and later provides a builtin
% \ifCLASSINFOpdf conditional that works the same way.
% When switching from latex to pdflatex and vice-versa, the compiler may
% have to be run twice to clear warning/error messages.






% *** CITATION PACKAGES ***
%
%\usepackage{cite}
% cite.sty was written by Donald Arseneau
% V1.6 and later of IEEEtran pre-defines the format of the cite.sty package
% \cite{} output to follow that of IEEE. Loading the cite package will
% result in citation numbers being automatically sorted and properly
% "compressed/ranged". e.g., [1], [9], [2], [7], [5], [6] without using
% cite.sty will become [1], [2], [5]--[7], [9] using cite.sty. cite.sty's
% \cite will automatically add leading space, if needed. Use cite.sty's
% noadjust option (cite.sty V3.8 and later) if you want to turn this off.
% cite.sty is already installed on most LaTeX systems. Be sure and use
% version 4.0 (2003-05-27) and later if using hyperref.sty. cite.sty does
% not currently provide for hyperlinked citations.
% The latest version can be obtained at:
% http://www.ctan.org/tex-archive/macros/latex/contrib/cite/
% The documentation is contained in the cite.sty file itself.






% *** GRAPHICS RELATED PACKAGES ***
%
\ifCLASSINFOpdf
  % \usepackage[pdftex]{graphicx}
  % declare the path(s) where your graphic files are
  % \graphicspath{{../pdf/}{../jpeg/}}
  % and their extensions so you won't have to specify these with
  % every instance of \includegraphics
  % \DeclareGraphicsExtensions{.pdf,.jpeg,.png}
\else
  % or other class option (dvipsone, dvipdf, if not using dvips). graphicx
  % will default to the driver specified in the system graphics.cfg if no
  % driver is specified.
  % \usepackage[dvips]{graphicx}
  % declare the path(s) where your graphic files are
  % \graphicspath{{../eps/}}
  % and their extensions so you won't have to specify these with
  % every instance of \includegraphics
  % \DeclareGraphicsExtensions{.eps}
\fi
% graphicx was written by David Carlisle and Sebastian Rahtz. It is
% required if you want graphics, photos, etc. graphicx.sty is already
% installed on most LaTeX systems. The latest version and documentation can
% be obtained at: 
% http://www.ctan.org/tex-archive/macros/latex/required/graphics/
% Another good source of documentation is "Using Imported Graphics in
% LaTeX2e" by Keith Reckdahl which can be found as epslatex.ps or
% epslatex.pdf at: http://www.ctan.org/tex-archive/info/
%
% latex, and pdflatex in dvi mode, support graphics in encapsulated
% postscript (.eps) format. pdflatex in pdf mode supports graphics
% in .pdf, .jpeg, .png and .mps (metapost) formats. Users should ensure
% that all non-photo figures use a vector format (.eps, .pdf, .mps) and
% not a bitmapped formats (.jpeg, .png). IEEE frowns on bitmapped formats
% which can result in "jaggedy"/blurry rendering of lines and letters as
% well as large increases in file sizes.
%
% You can find documentation about the pdfTeX application at:
% http://www.tug.org/applications/pdftex





% *** MATH PACKAGES ***
%
%\usepackage[cmex10]{amsmath}
% A popular package from the American Mathematical Society that provides
% many useful and powerful commands for dealing with mathematics. If using
% it, be sure to load this package with the cmex10 option to ensure that
% only type 1 fonts will utilized at all point sizes. Without this option,
% it is possible that some math symbols, particularly those within
% footnotes, will be rendered in bitmap form which will result in a
% document that can not be IEEE Xplore compliant!
%
% Also, note that the amsmath package sets \interdisplaylinepenalty to 10000
% thus preventing page breaks from occurring within multiline equations. Use:
%\interdisplaylinepenalty=2500
% after loading amsmath to restore such page breaks as IEEEtran.cls normally
% does. amsmath.sty is already installed on most LaTeX systems. The latest
% version and documentation can be obtained at:
% http://www.ctan.org/tex-archive/macros/latex/required/amslatex/math/





% *** SPECIALIZED LIST PACKAGES ***
%
%\usepackage{algorithmic}
% algorithmic.sty was written by Peter Williams and Rogerio Brito.
% This package provides an algorithmic environment fo describing algorithms.
% You can use the algorithmic environment in-text or within a figure
% environment to provide for a floating algorithm. Do NOT use the algorithm
% floating environment provided by algorithm.sty (by the same authors) or
% algorithm2e.sty (by Christophe Fiorio) as IEEE does not use dedicated
% algorithm float types and packages that provide these will not provide
% correct IEEE style captions. The latest version and documentation of
% algorithmic.sty can be obtained at:
% http://www.ctan.org/tex-archive/macros/latex/contrib/algorithms/
% There is also a support site at:
% http://algorithms.berlios.de/index.html
% Also of interest may be the (relatively newer and more customizable)
% algorithmicx.sty package by Szasz Janos:
% http://www.ctan.org/tex-archive/macros/latex/contrib/algorithmicx/




% *** ALIGNMENT PACKAGES ***
%
%\usepackage{array}
% Frank Mittelbach's and David Carlisle's array.sty patches and improves
% the standard LaTeX2e array and tabular environments to provide better
% appearance and additional user controls. As the default LaTeX2e table
% generation code is lacking to the point of almost being broken with
% respect to the quality of the end results, all users are strongly
% advised to use an enhanced (at the very least that provided by array.sty)
% set of table tools. array.sty is already installed on most systems. The
% latest version and documentation can be obtained at:
% http://www.ctan.org/tex-archive/macros/latex/required/tools/


%\usepackage{mdwmath}
%\usepackage{mdwtab}
% Also highly recommended is Mark Wooding's extremely powerful MDW tools,
% especially mdwmath.sty and mdwtab.sty which are used to format equations
% and tables, respectively. The MDWtools set is already installed on most
% LaTeX systems. The lastest version and documentation is available at:
% http://www.ctan.org/tex-archive/macros/latex/contrib/mdwtools/


% IEEEtran contains the IEEEeqnarray family of commands that can be used to
% generate multiline equations as well as matrices, tables, etc., of high
% quality.


%\usepackage{eqparbox}
% Also of notable interest is Scott Pakin's eqparbox package for creating
% (automatically sized) equal width boxes - aka "natural width parboxes".
% Available at:
% http://www.ctan.org/tex-archive/macros/latex/contrib/eqparbox/





% *** SUBFIGURE PACKAGES ***
%\usepackage[tight,footnotesize]{subfigure}
% subfigure.sty was written by Steven Douglas Cochran. This package makes it
% easy to put subfigures in your figures. e.g., "Figure 1a and 1b". For IEEE
% work, it is a good idea to load it with the tight package option to reduce
% the amount of white space around the subfigures. subfigure.sty is already
% installed on most LaTeX systems. The latest version and documentation can
% be obtained at:
% http://www.ctan.org/tex-archive/obsolete/macros/latex/contrib/subfigure/
% subfigure.sty has been superceeded by subfig.sty.



%\usepackage[caption=false]{caption}
%\usepackage[font=footnotesize]{subfig}
% subfig.sty, also written by Steven Douglas Cochran, is the modern
% replacement for subfigure.sty. However, subfig.sty requires and
% automatically loads Axel Sommerfeldt's caption.sty which will override
% IEEEtran.cls handling of captions and this will result in nonIEEE style
% figure/table captions. To prevent this problem, be sure and preload
% caption.sty with its "caption=false" package option. This is will preserve
% IEEEtran.cls handing of captions. Version 1.3 (2005/06/28) and later 
% (recommended due to many improvements over 1.2) of subfig.sty supports
% the caption=false option directly:
%\usepackage[caption=false,font=footnotesize]{subfig}
%
% The latest version and documentation can be obtained at:
% http://www.ctan.org/tex-archive/macros/latex/contrib/subfig/
% The latest version and documentation of caption.sty can be obtained at:
% http://www.ctan.org/tex-archive/macros/latex/contrib/caption/




% *** FLOAT PACKAGES ***
%
%\usepackage{fixltx2e}
% fixltx2e, the successor to the earlier fix2col.sty, was written by
% Frank Mittelbach and David Carlisle. This package corrects a few problems
% in the LaTeX2e kernel, the most notable of which is that in current
% LaTeX2e releases, the ordering of single and double column floats is not
% guaranteed to be preserved. Thus, an unpatched LaTeX2e can allow a
% single column figure to be placed prior to an earlier double column
% figure. The latest version and documentation can be found at:
% http://www.ctan.org/tex-archive/macros/latex/base/



%\usepackage{stfloats}
% stfloats.sty was written by Sigitas Tolusis. This package gives LaTeX2e
% the ability to do double column floats at the bottom of the page as well
% as the top. (e.g., "\begin{figure*}[!b]" is not normally possible in
% LaTeX2e). It also provides a command:
%\fnbelowfloat
% to enable the placement of footnotes below bottom floats (the standard
% LaTeX2e kernel puts them above bottom floats). This is an invasive package
% which rewrites many portions of the LaTeX2e float routines. It may not work
% with other packages that modify the LaTeX2e float routines. The latest
% version and documentation can be obtained at:
% http://www.ctan.org/tex-archive/macros/latex/contrib/sttools/
% Documentation is contained in the stfloats.sty comments as well as in the
% presfull.pdf file. Do not use the stfloats baselinefloat ability as IEEE
% does not allow \baselineskip to stretch. Authors submitting work to the
% IEEE should note that IEEE rarely uses double column equations and
% that authors should try to avoid such use. Do not be tempted to use the
% cuted.sty or midfloat.sty packages (also by Sigitas Tolusis) as IEEE does
% not format its papers in such ways.





% *** PDF, URL AND HYPERLINK PACKAGES ***
%
%\usepackage{url}
% url.sty was written by Donald Arseneau. It provides better support for
% handling and breaking URLs. url.sty is already installed on most LaTeX
% systems. The latest version can be obtained at:
% http://www.ctan.org/tex-archive/macros/latex/contrib/misc/
% Read the url.sty source comments for usage information. Basically,
% \url{my_url_here}.





% *** Do not adjust lengths that control margins, column widths, etc. ***
% *** Do not use packages that alter fonts (such as pslatex).         ***
% There should be no need to do such things with IEEEtran.cls V1.6 and later.
% (Unless specifically asked to do so by the journal or conference you plan
% to submit to, of course. )


% correct bad hyphenation here
\hyphenation{op-tical net-works semi-conduc-tor}


\begin{document}
%
% paper title
% can use linebreaks \\ within to get better formatting as desired
\title{Steganography}


% author names and affiliations
% use a multiple column layout for up to two different
% affiliations

\author{\IEEEauthorblockN{Amirul Azuani Binti Romle}
\IEEEauthorblockA{Fakulti Teknologi dan Sains Maklumat (FTSM)\\
Universiti Kebangsaan Malaysia\\
Bangi, Malaysia\\
Email: amirazua@yahoo.com}

}

% conference papers do not typically use \thanks and this command
% is locked out in conference mode. If really needed, such as for
% the acknowledgment of grants, issue a \IEEEoverridecommandlockouts
% after \documentclass

% for over three affiliations, or if they all won't fit within the width
% of the page, use this alternative format:
% 
%\author{\IEEEauthorblockN{Michael Shell\IEEEauthorrefmark{1},
%Homer Simpson\IEEEauthorrefmark{2},
%James Kirk\IEEEauthorrefmark{3}, 
%Montgomery Scott\IEEEauthorrefmark{3} and
%Eldon Tyrell\IEEEauthorrefmark{4}}
%\IEEEauthorblockA{\IEEEauthorrefmark{1}School of Electrical and Computer Engineering\\
%Georgia Institute of Technology,
%Atlanta, Georgia 30332--0250\\ Email: see http://www.michaelshell.org/contact.html}
%\IEEEauthorblockA{\IEEEauthorrefmark{2}Twentieth Century Fox, Springfield, USA\\
%Email: homer@thesimpsons.com}
%\IEEEauthorblockA{\IEEEauthorrefmark{3}Starfleet Academy, San Francisco, California 96678-2391\\
%Telephone: (800) 555--1212, Fax: (888) 555--1212}
%\IEEEauthorblockA{\IEEEauthorrefmark{4}Tyrell Inc., 123 Replicant Street, Los Angeles, California 90210--4321}}




% use for special paper notices
%\IEEEspecialpapernotice{(Invited Paper)}




% make the title area
\maketitle


\begin{abstract}
%\boldmath
Steganography is a branch of computer security that has been used for a long time ago since the ancient age. It has been used as a medium of communication between two parties who do not want anyone else other than themselves to know about the messages they are communicating. In other words they are communicating secret messages. In the past history, one of Steganography scheme is to shave the head of a messenger and tattoos the secret messages on the messenger's head. When the hairs regrown and covered the secret messages, the messenger can be sent to the intended recipient and the messenger's head will be shaved again to reveal the secret messsages. Steganography differs from crytopgraphy in the sense of steganography makes the secret messages invisible and making it unsuspiciuos, so nobody even knows that there is encrypted information except for the intended recipient. Unlike steganography,cryptography encrypted messages is visible, so that no matter how unbreakable it will still provoke suspicion. Steganography has evolve into a digital strategy of hiding messages, particularly suitable to be used in large file such as image, audio or even video file.

\end{abstract}
% IEEEtran.cls defaults to using nonbold math in the Abstract.
% This preserves the distinction between vectors and scalars. However,
% if the conference you are submitting to favors bold math in the abstract,
% then you can use LaTeX's standard command \boldmath at the very start
% of the abstract to achieve this. Many IEEE journals/conferences frown on
% math in the abstract anyway.

% no keywords




% For peer review papers, you can put extra information on the cover
% page as needed:
% \ifCLASSOPTIONpeerreview
% \begin{center} \bfseries EDICS Category: 3-BBND \end{center}
% \fi
%
% For peerreview papers, this IEEEtran command inserts a page break and
% creates the second title. It will be ignored for other modes.
\IEEEpeerreviewmaketitle



\section{Introduction}
Nowadays steganography has evolved from physical steganography used in past history to modern steganography, better known as digital steganography that entered the world in 1945. Development following that was slow, but has since taken off, going by the number of 'stego' programs : Over 725 digital steganography applications have been identified by the Steganography Analysis and Research Center.[1] Steganography is a tremendous method of embedding information of the content of one file within another file. It is the art of science that hide messages using digital strategy that hides files in some form of multimedia, such as image, audio or video file in such way that no one apart from the intended recipient knows the existence of the messages. The advantages of steganography over cryptography is that messages do not attract attention to themselves.Therefore, cryptography is protecting the contents of messages, while steganography protects both contents of messages and communicating parties. 


% no \IEEEPARstart

% You must have at least 2 lines in the paragraph with the drop letter
% (should never be an issue)


%\hfill 
 
%\hfill 

\section{How Steganography Works}
There are many methods that can be used to hide data inside image, video and audio files. However, the most commonly used methods are :

\begin{enumerate}
\item Least Significant Bytes\\
Each files that has been created usually contains some bytes that are not really used or are not very important. These kind of bytes can be used to hide information to be hidden by replacing it. This can be done without altering or damaging the file.
\\
\item Injection\\
This method is very simple and straight forward.The information to be hidden can directly injected into the file. However, there is drawback of using this method. It can increase the file size significantly, making the file suspicious.

\end{enumerate}

\section{Steganography and the Carrier Files}
\\
\subsection{Steganography in Images}
\\
The best type of images file that can be used for hiding information is a 24 Bit BMP (Bitmap)image. This is because, the 24 Bit BMP file is the largest type of file and usually is the highest quality. It is easier to hide information in a large file that contain of high quality and resolution.However, some people tend to choose smaller size file like 8 Bit BMP or other type of image format for instance the GIF file. The reason is posting a large size image file may invoke suspicious. It is good to know that a hidden information within an image file will be lost if the file is converted to other type of image file.
\\
\subsection{Steganography in Audio}
\\
\\
There are three techniques that can be used to conceal information in an audio file. The first technique is called low bit encoding. It is similar to LSB technique that is used in image file. However, there is drawback using this technique since it always noticeable to human ear. Therefore it is risky to use this technique to hide information within an audio file. The second technique is Spread Spectrum. It is used by adding random noises to the signal the information is conceal inside an audio file and spread across frequency spectrum. The third technique is Echo Data Hiding. This technique uses echo in the audio file to hide information. It works by adding extra sound to an echo inside an audio file. The advantages of using this technique is it can improve the sound of an audio inside the audio file.
\\
\subsection{Steganography in Video}
\\
\\
The method for Steganography in video usually is using the DCT (Discrete Cosine Transform) method. DCT works by altering certain parts of images in the video, and most commonly it will round them up. For instance the image has value of 5.667 it will round it up to 6. Most of the time it is not noticeable by human's eye. However, Steganography in video is the same as Steganography in images, the more information hidden in it, the more noticeable it will become.
\\
\subsection{Steganography in Documents}
\\
\\
Steganography also can be used in documents. It is very simple as just adding white space and tabs to the end of the lines of a document. Steganography in documents is very effective since the use of white space and tabs is not visible to human's eye at all. White space and tabs often occur naturally in documents, so that there is impossible to be suspicious to the file.

\section{How to Detect Steganography}\\
\\
\\
Steganography can be detected by using Steganalysis. It works by sensing whether a file contain hidden information or not. Though, Steganalysis does not decrypt the hidden information. There are two methods that can be used for Steganalysis. The methods are:
\\
\begin{itemize}
\item View and Compare Suspected File with Another Copy of the   		  File\\
      It works by observing two files whether they looks  				  identical. Look also at the file size, if one is larger 			  than the other, most probably the suspected file contains 		  hidden information.\\
\item Listen to the File\\
      This method is used in audio file. It works by just   			  listening to two same audio files.If any different sound  		  detected, might be the suspected file contains hidden 			  information.              
\end{itemize}
\\
\section{Steganography Tools}\\
\\
There are a lot of steganography tools available nowadays. Steganography tools are almost software that can be used to hide information within a file.\\
\begin{itemize}
\item JPHIDE and JPSEEK\\
      JPHIDE and JPSEEK are programs that allow information to be 	  hide in a jpeg visual image file. JPHIDE and JPSEEK is not 		  meant to hide information only, but it also hide 					  information that make it impossible to prove that the file 		  contains hidden information.\\
\item Blindside\\
	  Blindside works by concealing information, or set of files  	  within a standard computer image. The new image will looks 		  identical to human's eye. The hidden information can 				  contains up to 50k.It also can include password encrypted 		  to prevent unauthorized access to the hidden information.\\
\item Gifshuffle\\
      Gifshuffle is specifically used to conceal information in 		  GIF images. It works by shuffling the colourmap. which 			  leaves the visible image unchanged. It can be used with all 	  GIF images, including transparency and animation images. In 	  addition, Gifshuffle provides compression and encryption of 	  the concealed information.\\
\item Data Marking Technologies\\
      Data Marking Technologies currently have four types digital 	  steganography product that is available in the market. The 		  first one is StegComm which is used for confidential 				  multimedia communication. The second is StegMark which is   	  used for digital watermarking of digital storage media. The 	  third is StegSafe which is used for digital storage and 			  linkage and the fourth is StegSign which is used for e-			  commerce transactions.\\
\item Digital Picture Envelope\\
      Digital Picture Envelope can hide information so that it is 	  hardly notice by human's eye. Thus, hidden information can 		  be very safely stored or send inside the computer or to be 		  sent to any recipient. It works by embedding the hidden 			  information in a container image without changing the 			  visual quality of the container image or even the file size 	  is remain unchanged. To reveal the hidden information, it 		  can be restore from the secret-embedded container image.\\
\item Invisible Secrets\\
      Invisible Secrets will conceal the information to be hidden 	  as a naive looking files such as pictures or web pages. It 		  also include several interesting. The features are strong 		  encryption algorithm, locker that allows password to 				  protect the application, manage and generate password, a 			  shredder that destroy beyond recovery files, folders and 			  internet traces, able to create self-decrypting packages 			  and secured password transfer.\\
\item Snow\\
	  Snow is used to conceal information in ASCII text. It works 	  by appending whitespace to the end of lines. Spaces and 			  tabs invisible from viewer, make it effective to hide 			  information from casual viewers. Even if the hidden 				  information is detected, it cannot be read if built-in 			  encryption is applied.\\
\item Steg Party\\
	  Steg Party is used to hide information inside of plain text 	  files. It works by making small alteration to the messages 		  for instances, changing the spelling or punctuation. Due to 	  that any plain text files can be used with this software 			  and the hidden message will be more-or-less understandable 		  when it is embedded.\\	  	             	  
\end{itemize}
\section{Steganalysis Tools}
Steganalysis tools focuses on detecting Steganography and destroying the original messages. It is difficult to discover and decrypting the messages since encryption keys need to be known. Generally, it is difficult to detect Steganography that applied intensively high compression rate.Despite the fact, it still possible due to high compression will reduces the amount of information to be hide, leads to raising in encoding density that makes easier detection. The tools that available for Steganalysis are :

\begin{itemize}
\item Stegdetect\\
	  Stegdetect is an automated tool that is capable of 				  detecting steganographic contents in images files. It is 			  able to detect steganographic contents that uses different 		  steganography tools to embed hidden information in JPEG 			  images files. Presently, the detectable schemes are JSteg, 		  JPHide, Invisible Secrets, Outguess 01.3b, F5 and appendX 		  and camouflage. Stegbreak is used to launch dictionary 			  attacks against JSteg-Shell, JPHide and OutGuess 0.13b.\\
\item Steganography Analyzer Artifact Scanner (StegAlyzerAS)\\
      Steganography Analyzer Artifact Scanner has the capability 		  to scan the entire system, or individual directories, on 			  suspect media for the presence of Steganography application 	  artifacts. It also able to perform an automated or manual 		  search of the Windows Registry to determine whether any of 		  the Registry key is associated with Steganography 				  application.\\
\item Steganography Analyzer Signature Scanner (StegAlyzerSS)\\
      Steganography Analyzer Signature Scanner has the capability 	  to scan every file on suspected media for the presence of 		  Steganography applications in the files such as hexadecimal 	  bytes patterns or signatures.If a known signature is 				  detected, it is possible to extract hidden information with 	  the Steganography application associated with the 				  signature.     	   
\end{itemize} 
\\
\section{Steganography and Security}\\
As discussed earlier, Steganography is an effective way to hide information, thus protecting it from unauthorized user or access.However, Steganography is just one of the method protect the confidentiality of data. Steganography will be best used in combination with another data hiding method. The combination will be a part of a layered security approach. Some of good data hiding method that can be used in conjunction with Steganography include :\\

\begin{itemize}
\item Encryption\\
      Encryption is the process of passing data or plaintext 			  through a series of mathematical operations that generate 		  another form of original data that is called ciphertext. 			  Only parties that have key to decrypt the ciphertext back 		  to the original data can read the encrypted data. However, 		  encryption will not hide data, but it will only make the 			  data difficult to read.\\
\item Hidden Directories\\ 
      Hidden Directories is a feature offers by windows, which 			  allows files to be hide. This feature is simple as just 			  changing the property of a file to 'hidden' and the file 			  will be invisible. Conversely, the file will be visible if 		  someone choose to displays all types of files in their 			  explorer.\\
\item Covert Channels\\
      Covert Channels is tools that can be used to transmit 			  important data in seemingly normal network traffic. The 			  example of the tool is called Loki. Loki hides data in ICMP 	  traffic.\\            
\end{itemize}
\\
% An example of a floating figure using the graphicx package.
% Note that \label must occur AFTER (or within) \caption.
% For figures, \caption should occur after the \includegraphics.
% Note that IEEEtran v1.7 and later has special internal code that
% is designed to preserve the operation of \label within \caption
% even when the captionsoff option is in effect. However, because
% of issues like this, it may be the safest practice to put all your
% \label just after \caption rather than within \caption{}.
%
% Reminder: the "draftcls" or "draftclsnofoot", not "draft", class
% option should be used if it is desired that the figures are to be
% displayed while in draft mode.
%
%\begin{figure}[!t]
%\centering
%\includegraphics[width=2.5in]{myfigure}
% where an .eps filename suffix will be assumed under latex, 
% and a .pdf suffix will be assumed for pdflatex; or what has been declared
% via \DeclareGraphicsExtensions.
%\caption{Simulation Results}
%\label{fig_sim}
%\end{figure}

% Note that IEEE typically puts floats only at the top, even when this
% results in a large percentage of a column being occupied by floats.


% An example of a double column floating figure using two subfigures.
% (The subfig.sty package must be loaded for this to work.)
% The subfigure \label commands are set within each subfloat command, the
% \label for the overall figure must come after \caption.
% \hfil must be used as a separator to get equal spacing.
% The subfigure.sty package works much the same way, except \subfigure is
% used instead of \subfloat.
%
%\begin{figure*}[!t]
%\centerline{\subfloat[Case I]\includegraphics[width=2.5in]{subfigcase1}%
%\label{fig_first_case}}
%\hfil
%\subfloat[Case II]{\includegraphics[width=2.5in]{subfigcase2}%
%\label{fig_second_case}}}
%\caption{Simulation results}
%\label{fig_sim}
%\end{figure*}
%
% Note that often IEEE papers with subfigures do not employ subfigure
% captions (using the optional argument to \subfloat), but instead will
% reference/describe all of them (a), (b), etc., within the main caption.


% An example of a floating table. Note that, for IEEE style tables, the 
% \caption command should come BEFORE the table. Table text will default to
% \footnotesize as IEEE normally uses this smaller font for tables.
% The \label must come after \caption as always.
%
%\begin{table}[!t]
%% increase table row spacing, adjust to taste
%\renewcommand{\arraystretch}{1.3}
% if using array.sty, it might be a good idea to tweak the value of
% \extrarowheight as needed to properly center the text within the cells
%\caption{An Example of a Table}
%\label{table_example}
%\centering
%% Some packages, such as MDW tools, offer better commands for making tables
%% than the plain LaTeX2e tabular which is used here.
%\begin{tabular}{|c||c|}
%\hline
%One & Two\\
%\hline
%Three & Four\\
%\hline
%\end{tabular}
%\end{table}


% Note that IEEE does not put floats in the very first column - or typically
% anywhere on the first page for that matter. Also, in-text middle ("here")
% positioning is not used. Most IEEE journals/conferences use top floats
% exclusively. Note that, LaTeX2e, unlike IEEE journals/conferences, places
% footnotes above bottom floats. This can be corrected via the \fnbelowfloat
% command of the stfloats package.



\section{Conclusion}
Steganography is an old and incredibly versatile and effective technique for obscuring information. The idea of Steganography has been there for a long time and now it has been evolve into modern Steganography that has many method and tools to accomplished Steganography. Although methods of detecting Steganography known as Steganalysis exists, they cannot be entirely relied upon, as none of the techniques is 100 percent effective. Any attempt to detect and thwart the Steganography must use combination of detection methods and vigilance to blunt the effectiveness of the process.

% conference papers do not normally have an appendix


% use section* for acknowledgement
\section*{Acknowledgment}
I would like to express my gratitude to Mr. Mohd Zamri Murah the lecturer of Client Server for his precious guidance, all my colleague for their companionship and last but not least to my husband for his endless support.




% trigger a \newpage just before the given reference
% number - used to balance the columns on the last page
% adjust value as needed - may need to be readjusted if
% the document is modified later
%\IEEEtriggeratref{8}
% The "triggered" command can be changed if desired:
%\IEEEtriggercmd{\enlargethispage{-5in}}

% references section

% can use a bibliography generated by BibTeX as a .bbl file
% BibTeX documentation can be easily obtained at:
% http://www.ctan.org/tex-archive/biblio/bibtex/contrib/doc/
% The IEEEtran BibTeX style support page is at:
% http://www.michaelshell.org/tex/ieeetran/bibtex/
%\bibliographystyle{IEEEtran}
% argument is your BibTeX string definitions and bibliography database(s)
%\bibliography{IEEEabrv,../bib/paper}
%
% <OR> manually copy in the resultant .bbl file
% set second argument of \begin to the number of references
% (used to reserve space for the reference number labels box)
\begin{thebibliography} {99}
\bibitem{1}
 \emph{Steganography }, Wikimedia Foundation Inc..\hskip 1em plus
  0.5em minus 0.4em\relax 23rd August 2009, 16th August 2009:http://en.wikipedia.org/Wiki/Steganography

\bibitem{IEEEhowto:kopka}
 \emph{A Guide to \LaTeX}, 3rd~ed.\hskip 1em plus
  0.5em minus 0.4em\relax Harlow, England: Addison-Wesley, 1999.
\end{thebibliography}





% that's all folks
\end{document}


